\section{Kendte uendelige rækker}
Følgende er eksempler på kendte uendelige rækker.

\subsection{Den geometriske række}
\underline{\hyperref[afs:georæk]{Den Geometriske Række}} foreskriver at $a, z \in \mathbb{C}$, for
\[ 
\sum_{n = 0}^{\infty}  a z^n 
.\]
Denne har summen
\[ 
\sum_{n = 0}^{\infty} a z^{n} = \frac{a}{1-z}
.\]


\subsection{Den harmoniske række}
\underline{\hyperref[afs:harræk]{Den Harmoniske Række}} er defineret som
\[ 
\sum_{n = 1}^{\infty} \frac{1}{n}
.\]
Denne række er divergent. Faktisk er den et eksempel på den mere generelle
\[ 
\sum_{n = 1}^{\infty} \frac{1}{n^{p}}
\]
som er konvergent for $p > 1$ og divergent for $p \leq 1$.

\subsection{Basel-problemet}
\underline{\hyperref[afs:baspro]{Basel-problemet}} er defineret som
\[ 
\sum_{n = 1}^{\infty} \frac{1}{n^2}
.\]
Denne er ligeledes et eksempel på en række på formen
\[ 
\sum_{n = 1}^{\infty} \frac{1}{n^{p}}
\]
med $p = 2$ og den er derfor konvergent. Faktisk er summen
\[ 
\sum_{n = 1}^{\infty} \frac{1}{n^2} = \frac{\pi^2}{6}
.\]
