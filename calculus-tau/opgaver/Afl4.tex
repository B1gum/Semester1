\documentclass[12pt]{article}

\usepackage[utf8]{inputenc}
\usepackage[danish]{babel}
\usepackage{latexsym, amsfonts, amssymb, amsthm, amsmath, siunitx, graphicx, pgfplots}

\usepackage{sagetex}

\usepackage{import}
\usepackage{pdfpages}
\usepackage{transparent}
\usepackage{xcolor}

\newcommand{\incfig}[2][1]{%
    \def\svgwidth{#1\columnwidth}
    \import{./figures/}{#2.pdf_tex}
}

\pdfsuppresswarningpagegroup=1

\setlength{\parindent}{0in}
\setlength{\oddsidemargin}{0in}
\setlength{\textwidth}{6.5in}
\setlength{\textheight}{8.8in}
\setlength{\topmargin}{0in}
\setlength{\headheight}{18pt}

\pgfplotsset{compat=newest}

\pgfplotsset{every axis/.append style={
		axis x line=middle,    % put the x axis in the middle
		axis y line=middle,    % put the y axis in the middle
		axis line style={<->,color=black}, % arrows on the axis
	}}

\title{Afleveringsopgave uge 4}
\author{Noah Rahbek Bigum Hansen}
\date{22. / 9. - 2024}

\begin{document}

\maketitle

\section*{Opgave 4.18}

Givet ligningen:
\[
2z^2+4z+8=0
.\] 
Ønskes fundet begge de komplekse løsnigner. Derudover ønskes at finde den polære form for de komplekse løsninger. \\
For andengradsligniner gælder at de har to komplekse løsninger er givet ved:
	\[
		z = \frac{-b \pm \sqrt{b^2-4ac}}{2a} 
	.\] 
Først simplificeres den givne ligning ved division med 2:
		\[
		z^2+2+4=0
		.\] 

Koefficienterne fra denne givne ligning indsættes i løsningsformlen for andengradsligniner:
	\[
		z = \frac{-2 \pm \sqrt{2^2-4\cdot 4}}{2} = -1 \pm \frac{i\sqrt{12}}{2} 	.\] 
2.-leddet simplificeres:
	\[
	z = -1 \pm \frac{2i\sqrt{3}}{2} = -1 \pm i\sqrt{3} 
	.\]
Den absolutte værdi findes:
	\[
	|z| = \sqrt{1^2 + (\sqrt{3})^2} = \sqrt{4} = 2 
	.\] 
Og dernæst skal argumentet findes. Det kan hurtigt ses at de to løsninger ligger i hhv. 2.- og 3.-kvadrant i et koordinatsystem. Her vil blot findes argumentet til løsningen i 2.-kvadrant idet argumentet til løsningen i 3.kvadrant blot er løsningen i 2.-kvadrants multipliceret med $-1$. For at finde løsningen omtalt før starter vi med at 'dreje' koordinatsystemet 90 grader, hvilket der tages højde for ved at addere det endelige resultet med  $\frac{1}{2}\pi$:
	\[
	\arg(z) = \pm \frac{1}{2}\pi + \alpha
	.\] 
Denne vinkel, $\alpha$ kan findes vha. simpel trignometri, idet vi har:
	\[
		|z|\sin{\alpha} = 1
	.\] 
Altså har vi:
	\[
		\alpha = \sin^{-1}\left( \frac{1}{2} \right) = \frac{\pi}{6}
	.\] 
Og dermed fås:
	\[
	\arg(z) = \pm \frac{2}{3}\pi
	.\] 
Nu er opgaven streng talt løst, men for god ordens skyld opskrives løsningerne også på polær form:
	\[
	z = 2e^{\pm \left( \frac{2}{3} i \pi \right) }
	.\] 
\end{document}
