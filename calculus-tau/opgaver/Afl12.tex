\documentclass[12pt]{article}
\usepackage[danish]{babel}
\usepackage{amsfonts, amssymb, mathtools, amsthm, amsmath}
\usepackage{graphicx, pgfplots}
\usepackage{url}
\usepackage[dvipsnames]{xcolor}
\usepackage{sagetex}
\usepackage{lastpage}

%loaded last
\usepackage[hidelinks]{hyperref}

\usepackage{siunitx}
  \sisetup{exponent-product = \cdot,
    output-decimal-marker = {,}}

%Giles Castelles incfig
\usepackage{import}
\usepackage{xifthen}
\usepackage{pdfpages}
\usepackage{transparent}

\newcommand{\incfig}[2][1]{%
  \def\svgwidth{#1\columnwidth}
  \import{../figures/}{#2.pdf_tex}
}

\setlength{\parindent}{0in}
\setlength{\oddsidemargin}{0in}
\setlength{\textwidth}{6.5in}
\setlength{\textheight}{8.8in}
\setlength{\topmargin}{0in}
\setlength{\headheight}{18pt}

\usepackage{fancyhdr}
\pagestyle{fancy}

\fancyhead{}
\fancyfoot{}
\fancyfoot[R]{\thepage}
\fancyhead[C]{\leftmark}

\pgfplotsset{compat=newest}

\pgfplotsset{every axis/.append style={
  axis x line=middle,    % put the x axis in the middle
  axis y line=middle,    % put the y axis in the middle
  axis line style={<->,color=black}, % arrows on the axis
}}

\usepackage{thmtools}
\usepackage{tcolorbox}
  \tcbuselibrary{skins, breakable}
  \tcbset{
    space to upper=1em,
    space to lower=1em,
  }

\theoremstyle{definition}

\newtcolorbox[auto counter]{definition}[1][]{%
  breakable,
  colframe=ForestGreen,  %frame color
  colback=ForestGreen!5, %background color
  colbacktitle=ForestGreen!25, %background color for title
  coltitle=ForestGreen!70!black,  %title color
  fonttitle=\bfseries\sffamily, %title font
  left=1em,              %space on left side in box,
  enhanced,              %more options
  frame hidden,          %hide frame
  borderline west={2pt}{0pt}{ForestGreen},  %display left line
  title=Definition \thetcbcounter: #1,
}

\newtcolorbox{greenline}{%
  breakable,
  colframe=ForestGreen,  %frame color
  colback=white,          %remove background color
  left=1em,              %space on left side in box
  enhanced,              %more options
  frame hidden,          %hide frame
  borderline west={2pt}{0pt}{ForestGreen},  %display left line
}

\newtcolorbox[auto counter, number within=section]{eks}[1][]{%
  brekable,
  colframe=NavyBlue,  %frame color
  colback=NavyBlue!5, %background color
  colbacktitle=NavyBlue!25,    %background color for title
  coltitle=NavyBlue!70!black,  %title color
  fonttitle=\bfseries\sffamily, %title font
  left=1em,            %space on left side in box,
  enhanced,            %more options
  frame hidden,        %hide frame
  borderline west={2pt}{0pt}{NavyBlue},  %display left line
  title=Eksempel \thetcbcounter: #1
}

\newtcolorbox{blueline}{%
  breakable,
  colframe=NavyBlue,     %frame color
  colback=white,         %remove background
  left=1em,              %space on left side in box,
  enhanced,              %more options
  frame hidden,          %hide frame
  borderline west={2pt}{0pt}{NavyBlue},  %display left line
}

\newtcolorbox{teo}[1][]{%
  breakable,
  colframe=RawSienna,  %frame color
  colback=RawSienna!5, %background color
  colbacktitle=RawSienna!25,    %background color for title
  coltitle=RawSienna!70!black,  %title color
  fonttitle=\bfseries\sffamily, %title font
  left=1em,              %space on left side in box,
  enhanced,              %more options
  frame hidden,          %hide frame
  borderline west={2pt}{0pt}{RawSienna},  %display left line
  title=Teori: #1,
}

\newtcolorbox[auto counter, number within=section]{sæt}[1][]{%
  breakable,
  colframe=RawSienna,  %frame color
  colback=RawSienna!5, %background color
  colbacktitle=RawSienna!25,    %background color for title
  coltitle=RawSienna!70!black,  %title color
  fonttitle=\bfseries\sffamily, %title font
  left=1em,              %space on left side in box,
  enhanced,              %more options
  frame hidden,          %hide frame
  borderline west={2pt}{0pt}{RawSienna},  %display left line
  title=Sætning \thetcbcounter: #1,
  before lower={\textbf{Bevis:}\par\vspace{0.5em}},
  colbacklower=RawSienna!25,
}

\newtcolorbox{redline}{%
  breakable,
  colframe=RawSienna,  %frame color
  colback=white,       %Remove background color
  left=1em,            %space on left side in box,
  enhanced,            %more options
  frame hidden,        %hide frame
  borderline west={2pt}{0pt}{RawSienna},  %display left line
}

\newtcolorbox{for}[1][]{%
  breakable,
  colframe=NavyBlue,  %frame color
  colback=NavyBlue!5, %background color
  colbacktitle=NavyBlue!25,    %background color for title
  coltitle=NavyBlue!70!black,  %title color
  fonttitle=\bfseries\sffamily, %title font
  left=1em,              %space on left side in box,
  enhanced,              %more options
  frame hidden,          %hide frame
  borderline west={2pt}{0pt}{NavyBlue},  %display left line
  title=Forklaring #1,
}

\newtcolorbox{bem}{%
  breakable,
  colframe=NavyBlue,  %frame color
  colback=NavyBlue!5, %background color
  colbacktitle=NavyBlue!25,    %background color for title
  coltitle=NavyBlue!70!black,  %title color
  fonttitle=\bfseries\sffamily, %title font
  left=1em,              %space on left side in box,
  enhanced,              %more options
  frame hidden,          %hide frame
  borderline west={2pt}{0pt}{NavyBlue},  %display left line
  title=Bemærkning:,
}

\makeatother
\def\@lecture{}%
\newcommand{\lecture}[3]{
  \ifthenelse{\isempty{#3}}{%
    \def\@lecture{Lecture #1}%
  }{%
    \def\@lecture{Lecture #1: #3}%
  }%
  \subsection*{\makebox[\textwidth][l]{\@lecture \hfill \normalfont\small\textsf{#2}}}
}

\makeatletter

\newcommand{\opgave}[1]{%
 \def\@opgave{#1}%
 \subsection*{Opgave #1}
}

\makeatother

%Format lim the same way in intext and in display
\let\svlim\lim\def\lim{\svlim\limits}

% horizontal rule
\newcommand\hr{
\noindent\rule[0.5ex]{\linewidth}{0.5pt}
}

\title{Afleveringsopgave uge 12}
\author{Noah Rahbek Bigum Hansen}
\date{14. November 2024}

\begin{document}

\maketitle

\section*{Opg. 1.39}
If $E[X] = 3$ and $\mathrm{Var}(x) = 1$, find

\subsection*{(a)}
\[ 
  E \left[ (4X-1)^2 \right]
\]
\bigbreak
Først udvides argumentet til middelværdi-operatoren som
\[ 
  (4X-1)^2 = 16X^2 - 8X + 1
.\]
Vi kan dernæst omskrive udtrykket fra opgaven til
\[ 
  E \left[ (4X-1)^2 \right] = 16E \left[X^2 \right] - 8E[X] + 1
.\]
Vi har brug for at finde $E \left[ X^2 \right]$ for at kunne komme videre. For at finde værdien heraf bruger vi at variansen er givet som
\[
  \mathrm{Var}(X) = E \left[ X^2 \right] - E[X]^2
.\]
Sættes kendte størrelser ind i formlen fås at
\begin{align*}
  1 &= E \left[ X^2 \right] - 3^2 \\
  E \left[ X^2 \right] &= 10
.\end{align*}
Og da bliver det trivielt at løse udtrykket fra opgaven som
\[ 
  E \left[ (4X-1)^2 \right] = 16E \left[ X^2 \right] - 8E[X] + 1 = 16 \cdot 10 - 8\cdot 3 + 1 = 137
.\]
Altså er svaret 137.

\subsection*{(b)}
\[ 
  \mathrm{Var}(5-2X)
\]
\bigbreak
Vi bruger at følgende sammenhæng gælder for variansen
\[ 
  \mathrm{Var}(aX + b) = a^2\mathrm{Var}(X)
\]
og får at
\[ 
  \mathrm{Var}(5-2X) = (-2)^2\cdot \mathrm{Var}(X) = 4\cdot 1 = 4
.\]
Altså er svaret 4.



\section*{Opg. D}
To typer antibiotika bliver sammenlignet mod mellemørebetændelse. Det ene (Aoxicillin) et billigt, men ikke så effektivt. Det andet (Cefalor) er effektivt, men også lidt dyrere. I det følgende vil vi regne på hvilken behandling der er at foretrække ud fra et økonomisk synspunkt. Det oplyses at:

\vspace{14pt}

For Amoxicillin er der sandsynlighed \num{0,75} for at blive rask. Hvis patienten bliver rask af behandlingen, er udgiften 60 kr., hvorimod hvis patienten ikke bliver rask af behandlingen er de samlede udgifter 96 kr.

\vspace{14pt}

For Cefalor er der sandsynlighed \num{0,9} for at blive rask. Hvis patienten bliver rask af behandlingen, er udgiften 70 kr., hvorimod hvis patienten ikke bliver rask af behandlingen er de samlede omkostninger 106 kr.

\vspace{14pt}

Udregn middelomkostningerne for de to typer af antibiotika Amoxicillin og Cefalor.  Hvilken type skal vi vælge, hvis vi vil minimere udgifterne?
\bigbreak
Middelomkostningerne, $E[X]$, forbundet med en given type medicin må være det sandsynlighedsvægtede gennemsnit af omkostningerne så
\[ 
  E[X] = P(\mathrm{rask}) \cdot X_{rask} + P(\mathrm{syg}) \cdot X_{syg}
.\]
For Aoxicillin er dette
\[ 
  E[A] = \num{0,75} \cdot 60 + (1-\num{0,75}) \cdot 96 = 69 
.\]
Altså er den forventede omkostning for Aoxicillin-kuren 69 kr. Det samme kan gøres for Cefalor, med samme fremgangsmåde, som
\[ 
  E[C] = \num{0,9} \cdot 70 + (1-\num{0,9}) \cdot 106 = \num{73,6} 
.\]
Den forventede omkostning for Cefalor-kuren må da være \num{73,6} kr. Derfor må det også gælde at Aoxicillin kuren er bedst ift. at minimere udgifterne idet $E[A] = 69 < E[C] = \num{73,6}$.


\end{document}
