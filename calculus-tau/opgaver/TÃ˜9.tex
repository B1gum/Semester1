\documentclass[12pt]{article}
\usepackage[danish]{babel}
\usepackage{amsfonts, amssymb, mathtools, amsthm, amsmath}
\usepackage{graphicx, pgfplots}
\usepackage{url}
\usepackage[dvipsnames]{xcolor}
\usepackage{sagetex}

%loaded last
\usepackage[hidelinks]{hyperref}

\usepackage{siunitx}
  \sisetup{exponent-product = \cdot,
    output-decimal-marker = {,}}

%Giles Castelles incfig
\usepackage{import}
\usepackage{xifthen}
\usepackage{pdfpages}
\usepackage{transparent}

\newcommand{\incfig}[2][1]{%
  \def\svgwidth{#1\columnwidth}
  \import{../figures/}{#2.pdf_tex}
}

\setlength{\parindent}{0in}
\setlength{\oddsidemargin}{0in}
\setlength{\textwidth}{6.5in}
\setlength{\textheight}{8.8in}
\setlength{\topmargin}{0in}
\setlength{\headheight}{18pt}

\usepackage{fancyhdr}
\pagestyle{fancy}

\fancyhead{}
\fancyfoot{}
\fancyfoot[R]{\thepage}
\fancyhead[C]{\leftmark}

\pgfplotsset{compat=newest}

\pgfplotsset{every axis/.append style={
  axis x line=middle,    % put the x axis in the middle
  axis y line=middle,    % put the y axis in the middle
  axis line style={<->,color=black}, % arrows on the axis
}}

\usepackage{thmtools}
\usepackage{tcolorbox}
  \tcbuselibrary{skins, breakable}
  \tcbset{
    space to upper=1em,
    space to lower=1em,
  }

\theoremstyle{definition}

\newtcolorbox[auto counter]{definition}[1][]{%
  breakable,
  colframe=ForestGreen,  %frame color
  colback=ForestGreen!5, %background color
  colbacktitle=ForestGreen!25, %background color for title
  coltitle=ForestGreen!70!black,  %title color
  fonttitle=\bfseries\sffamily, %title font
  left=1em,              %space on left side in box,
  enhanced,              %more options
  frame hidden,          %hide frame
  borderline west={2pt}{0pt}{ForestGreen},  %display left line
  title=Definition \thetcbcounter: #1,
}

\newtcolorbox{greenline}{%
  breakable,
  colframe=ForestGreen,  %frame color
  colback=white,          %remove background color
  left=1em,              %space on left side in box
  enhanced,              %more options
  frame hidden,          %hide frame
  borderline west={2pt}{0pt}{ForestGreen},  %display left line
}

\newtcolorbox[auto counter, number within=section]{eks}[1][]{%
  brekable,
  colframe=NavyBlue,  %frame color
  colback=NavyBlue!5, %background color
  colbacktitle=NavyBlue!25,    %background color for title
  coltitle=NavyBlue!70!black,  %title color
  fonttitle=\bfseries\sffamily, %title font
  left=1em,            %space on left side in box,
  enhanced,            %more options
  frame hidden,        %hide frame
  borderline west={2pt}{0pt}{NavyBlue},  %display left line
  title=Eksempel \thetcbcounter: #1
}

\newtcolorbox{blueline}{%
  breakable,
  colframe=NavyBlue,     %frame color
  colback=white,         %remove background
  left=1em,              %space on left side in box,
  enhanced,              %more options
  frame hidden,          %hide frame
  borderline west={2pt}{0pt}{NavyBlue},  %display left line
}

\newtcolorbox{teo}[1][]{%
  breakable,
  colframe=RawSienna,  %frame color
  colback=RawSienna!5, %background color
  colbacktitle=RawSienna!25,    %background color for title
  coltitle=RawSienna!70!black,  %title color
  fonttitle=\bfseries\sffamily, %title font
  left=1em,              %space on left side in box,
  enhanced,              %more options
  frame hidden,          %hide frame
  borderline west={2pt}{0pt}{RawSienna},  %display left line
  title=Teori: #1,
}

\newtcolorbox[auto counter, number within=section]{sæt}[1][]{%
  breakable,
  colframe=RawSienna,  %frame color
  colback=RawSienna!5, %background color
  colbacktitle=RawSienna!25,    %background color for title
  coltitle=RawSienna!70!black,  %title color
  fonttitle=\bfseries\sffamily, %title font
  left=1em,              %space on left side in box,
  enhanced,              %more options
  frame hidden,          %hide frame
  borderline west={2pt}{0pt}{RawSienna},  %display left line
  title=Sætning \thetcbcounter: #1,
  before lower={\textbf{Bevis:}\par\vspace{0.5em}},
  colbacklower=RawSienna!25,
}

\newtcolorbox{redline}{%
  breakable,
  colframe=RawSienna,  %frame color
  colback=white,       %Remove background color
  left=1em,            %space on left side in box,
  enhanced,            %more options
  frame hidden,        %hide frame
  borderline west={2pt}{0pt}{RawSienna},  %display left line
}

\newtcolorbox{for}[1][]{%
  breakable,
  colframe=NavyBlue,  %frame color
  colback=NavyBlue!5, %background color
  colbacktitle=NavyBlue!25,    %background color for title
  coltitle=NavyBlue!70!black,  %title color
  fonttitle=\bfseries\sffamily, %title font
  left=1em,              %space on left side in box,
  enhanced,              %more options
  frame hidden,          %hide frame
  borderline west={2pt}{0pt}{NavyBlue},  %display left line
  title=Forklaring #1,
}

\newtcolorbox{bem}{%
  breakable,
  colframe=NavyBlue,  %frame color
  colback=NavyBlue!5, %background color
  colbacktitle=NavyBlue!25,    %background color for title
  coltitle=NavyBlue!70!black,  %title color
  fonttitle=\bfseries\sffamily, %title font
  left=1em,              %space on left side in box,
  enhanced,              %more options
  frame hidden,          %hide frame
  borderline west={2pt}{0pt}{NavyBlue},  %display left line
  title=Bemærkning:,
}

\makeatother
\def\@lecture{}%
\newcommand{\lecture}[3]{
  \ifthenelse{\isempty{#3}}{%
    \def\@lecture{Lecture #1}%
  }{%
    \def\@lecture{Lecture #1: #3}%
  }%
  \subsection*{\makebox[\textwidth][l]{\@lecture \hfill \normalfont\small\textsf{#2}}}
}

\makeatletter

\newcommand{\opgave}[1]{%
 \def\@opgave{#1}%
 \subsection*{Opgave #1}
}

\makeatother

%Format lim the same way in intext and in display
\let\svlim\lim\def\lim{\svlim\limits}

% horizontal rule
\newcommand\hr{
\noindent\rule[0.5ex]{\linewidth}{0.5pt}
}

\title{TØ-opgaver til uge 9}
\author{Noah Rahbek Bigum Hansen}
\date{4. November 2024}

\begin{document}

\maketitle

\section*{Opg. 9.2.23}
Find all second-order partial derivatives for the following.
\[ 
R(x,y) = 4x^5 - 8x^4y^8 + 6x^5y^6
.\]
\bigbreak
For at finde de 2. ordens afledede må vi først finde de to 1. ordens afledede som
\begin{align*}
  R_x &= 20x^{4} - 32x^{3}y^{8} + 30x^{4}y^{6} \\
  R_y &= -64x^{4}y^{7} + 36x^{5}y^{5}
\end{align*}
Dermed har vi at
\begin{align*}
  R_{x x} &= 80x^{3} - 96x^2y^{8} + 120x^3y^{6} \\
  R_{yy} &= -448 x^{4}y^{6} + 180 x^{5}y^{4} \\
  R_{xy} = R_{yx} &= -256x^{3}y^{7} + 180x^{4}y^{5} 
\end{align*}



\section*{Prøveeksamensopgave 15}
Betragt funktionen
\[ 
f(x,y) = 6y^3-8x^2y + 16x^2 + 35
.\]
Beregn
\[ 
\frac{\partial^2 f}{\partial x \partial y} = - kx
.\]
Hvor $k$ er et helt tal mellem $0$ og $99$.
\bigbreak
Vi har at
\begin{align*}
  \frac{\partial f}{\partial x} &= -16xy + 32x \\
  \frac{\partial^2 f}{\partial x \partial y} &= -16x
\end{align*}
Altså er $k = 16$

\section*{Opg. 9.2.63}
The reaction to $x$ units of a drug $t$ hours after it was administered is given by
\[ 
R(x,t) = x^2(a-x) t^2e^{-t}
,\]
for $0 \leq x \leq a$ (where $a$ is a constant). Find the following.
\bigbreak
Først udvider vi parentesen så vi får at
\[ 
R(x, t) = x^2at^2e^{-t}-x^3t^2e^{-t}
.\]


\subsection*{(a)}
\[ 
\frac{\partial R}{\partial x}
.\]
\bigbreak
Vi får at
\[ 
\frac{\partial R}{\partial x} = 2xat^2e^{-t}-3x^2t^2e^{-t} = x(2a-3x)t^2e^{-t}
.\]


\subsection*{(b)}
\[ 
\frac{\partial R}{\partial t}
.\]
\bigbreak
Vi får at
\[ 
\frac{\partial R}{\partial t} = x^2(a-x)(2te^{-t}-t^2e^{-t})
.\]


\subsection*{(c)}
\[ 
\frac{\partial^2 R}{\partial^2 x}
.\]
\bigbreak
Vi får at
\[ 
\frac{\partial^2 R}{\partial^2 x} = 2at^2e^{-t} - 6xt^2e^{-t}
.\]


\subsection*{(d)}
\[ 
\frac{\partial^2 R}{\partial x \partial t}
.\]
\bigbreak
Vi får at
\[ 
\frac{\partial^2 R}{\partial x \partial t} = x(2a-3x)(2t-t^2)e^{-t}
.\]

\section*{Opg. 9.2.69}
The gravitational attraction $F$ on a body a distance $r$ from the center of the Earth, where $r$ is greater than the radius of Earth, is a function of its mass $m$ and the distance $r$ as follows:
\[ 
F = \frac{mgR^2}{r^2}
,\]
where $R$ is the radius of Earth and $g$ is the force of gravity.

\subsection*{(a)}
Find and interpret $F_m$ and $F_r$.
\bigbreak
\[ 
F_m = \frac{\partial F}{\partial m} = \frac{gR^2}{r^2}
.\]
Dette fortolkes som, at hvis afstanden, $r$, holdes konstant så vil kraften stige med ovenstående konstant.

\[ 
F_r = \frac{\partial F}{\partial r} = -\frac{2mgR^2}{r^3}
.\]
Hvis afstanden $r$ øges, men massen $m$ holdes konstant så vil kraften falde


\subsection*{(b)}
Show that $F_m>0$ and $F_r<0$. Why is this reasonable?
\bigbreak
Fordi vi ved at kraften stiger ved øget masse og falder ved øget afstand.

\section*{Opg. 9.3.2}
Find all points where the functions have any relative extrema. Identify any saddle points.
\[ 
f(x,y) = 3xy + 6y-5x
.\]
\bigbreak
Først findes $f_x$ og $f_y$ som
\begin{align*}
  f_x &= 3y - 5 \\
  f_y &= 3x + 6 
\end{align*}
Altså har vi et kritisk punkt for
\begin{align*}
  0 &= 3y - 5  & 0 &= 3x + 6\\
  y &= \frac{5}{3} & x &= -2\\
\end{align*}
Altså er vores eneste kritiske punkt $(-2, \frac{5}{3})$. Vi tjekker om dette er et saddelpunkt ved at finde diskriminanten $D = f_{x x}f_{yy} - f_{xy}^2$. Altså findes de dobbeltafledede
\begin{align*}
  f_{x x} &= 0 \\
  f_{yy} &= 0 \\
  f_{xy} &= 3 \\
  D &= -3^2 = -9 < 0
.\end{align*}
Altså er der ingen minima eller maksima men dog et saddelpunkt i $(-2, \frac{5}{3})$.

\section*{Prøveeksamensopgave 16}
Betragt funktionen
\[ 
f(x,y) = 5x^2 + 10y^2 - 5xy -9x
.\]
Det oplyses at $f$ har et globalt minimum i $(x_0, y_0)$. Bestem $y_0$ når
\[ 
y_0 = \frac{9}{k}
.\]
Hvor $k$ er et helt tal mellem $0$ og $99$.
\bigbreak
Først finder vi $f_x$ og $f_y$ som
\begin{align*}
  f_x &= 10x - 5y - 9 \\
  f_y &= 20y - 5x
\end{align*}
Altså har vi minimum i
\begin{align*}
  0 &= 10x - 5y -9 & 0 &= 20y - 5x \\
  x &= \frac{1}{2}y + \frac{9}{10} & 0 &= 20y - \frac{5}{2}y - \frac{9}{2} = \frac{35}{2}y -  \frac{9}{2} \\
\end{align*}
\[ 
y = \frac{9}{35}
.\]
Altså er $k = 35$.

\section*{Opg. 9.3.31}
Consider the function $f(x,y) = x^2(y+1)^2 + k(x+1)^2y^2$.

\subsection*{(a)}
For what values of $k$ is the point $(x,y) = (0,0)$ a critical point?
\bigbreak
Vi finder først $f_x$ og $f_y$ som
\[ 
f_x = 2x(y+1)^2 + 2k(x+1)y^2
\]
og
\[ 
f_y = 2x^2(y+1) + 2k(x+1)^2y
.\]
Et kritisk punkt er hvor $f_x = 0$ og $f_y = 0$. Altså sættes punktet ind og ser for hvilke $k$-værdier det går op. Altså har vi at
\begin{align*}
  0 &= 2\cdot0(0+1)^2 + 2k(0+1)0^2 \\
  0 &= 2\cdot0^2(0+1) + 2k(0+1)^2\cdot0
\end{align*}
Det gælder altså for alle værdier af $k$.


\subsection*{(b)}
For what values of $k$ is the point $(x,y) = (0,0)$ a relative minimum of the function?
\textit{Hint: Betragt $k=0$ som et specialtilfælde.}
\bigbreak
Punktet er et minimum hvis $D > 0$ og $f_{x x} > 0$. Vi er derfor nødsaget til at finde de dobbeltafledede som
\begin{align*}
  f_{x x} &= 2(y+1)^2+2ky^2 \\
  f_{yy} &= 2x^2 + 2k(x+1)^2 \\
  f_{xy} &= 4x(y+1) + 4k(x+1)y
\end{align*}
Altså har vi at
\begin{align*}
  f_{x x}(0,0) &= 2(0+1)^2 + 2k0^2 = 2 \\
  f_{yy}(0,0) &= 2\cdot 0^2 + 2k(0+1)^2 = 2k \\
  f_{xy}(0,0= &= 4\cdot0(0+1) + 4k(0+1)\cdot0
\end{align*}
Vi har dermed at
\[ 
4k > 0 \implies k > 0
.\]
Altså er det et globalt minimum for alle $k>0$. For $k=0$ er testen inkonklusiv. Derfor er vi nødsaget til selv at indsætte punktet. Derfor får vi at
\begin{align*}
  f(x,y) &= x^2(y+1)^2 \\
  f(0,0) &= 0
,\end{align*}
$x^2(y+1)^2 > 0$ for alle $(x,y)$. Derfor er den også et globalt minimum for $k = 0$. Altså er løsningsmængden $k \geq 0$


\section*{Opg. 9.3.36}
\textbf{Cost.} The total cost (in dollars) to produce $x$ units of electrical tape and $y$ units of packing tape is given by
\[ 
C(x,y) = 2x^2 + 2y^2 -3xy + 4x -94 y + 4200
.\]
Find the number of units of each kind of tape that should be produced so that the total cost is a minimum. Find the minimum total cost.


\end{document}
