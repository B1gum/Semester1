\section{Gravitation}

\begin{table}[ht]
\begin{tabular}{|l|l|l|}
\hline
\textbf{Givet}                   & \textbf{Ønsker at finde}           & \textbf{Relevante formler} \\ \hline
To masser, afstand               & Tyngdekraft                        & \ref{afs:newtyn}           \\ \hline
To masser, afstand               & Tyngde                             & \ref{afs:tyngde}           \\ \hline
Masse af objekt, radius af objekt & Tyngdeacceleration ved overflade & \ref{afs:tynacc} \\ \hline
To masser, to afstande           & Tyngdekraftens arbejde             & \ref{afs:tynarb}           \\ \hline
To masser, afstand               & Potentiel energi for tyngdekraften & \ref{afs:tynpot}           \\ \hline
Masse, kredsløbsradius           & Hastighed af sattelit              & \ref{afs:hassat}           \\ \hline
Kredsløbsradius, omløbshastighed & Periode for sattelit               & \ref{afs:persat}           \\ \hline
\end{tabular}
\end{table}

\subsection{Newtons tyngdelov (13.1)}

\subsubsection{Newtons tyngdelov}
Newtons tyngdelov siger at: ``Enhver partikel i universet tiltrækker enhver anden partikel med en kraft, der er direkte proportional med produktet af partikernes masser og omvendt proportional med kvadratet af afstanden mellem dem.''

\subsubsection{Matematisk formulering af Newtons tyngdelov} \label{afs:newtyn}
Matematisk kan Newtons tyngdelov formuleres som
\[ 
F_g = \frac{Gm_1m_2}{r^2}
.\]
Hvor $F_g$ er tyngdekraften, $G \approx \qty{6,67408e-11}{N.m^2.kg^{-2}} $ er gravitationskonstanten, $m_1$ og $m_2$ er masserne af de to legemer som påvirker hinanden gravitationelt og $r$ er afstanden mellem de to legemer.


\subsection{Tyngde (Eng: \textit{Weight}) (13.2)}

\subsubsection{Tyngden af et objekt ved jordens overflade} \label{afs:tyngde}
Tyngden $w$ af et objekt ved jordens overflade er lig tyngdekraften fra jorden på objektet. Altså
\[ 
w = F_g = \frac{Gm_Em}{R_E^2}
.\]
Hvor $m_E$ er jordens masse, $R_E$ er jordens radius og $m$ er massen af objektet, hvis tyngde man ønsker at finde.


\subsubsection{Tyngdeaccelerationen} \label{afs:tynacc}
Tyngdeaccelerationen ved jordens overflade $g$ kan findes som
\[ 
g = \frac{Gm_E}{R_E^2}
.\]
Hvor $m_E$ er jordens masse og $R_E$ er jordens radius. 


\subsection{Potentiel energi i et tyngdefelt (13.3)}

\subsubsection{Tyngdekraftens arbejde} \label{afs:tynarb}
Tyngdekraftens arbejde $W_{\text{grav}}$ kan findes som
\[ 
W_{\text{grav}} = -GMm \int_{r_1}^{r_2} \frac{\mathrm{d}r}{r^2} = \frac{GMm}{r_2} - \frac{GMm}{r_1} 
.\]
Hvor $M$ og $m$ er masserne af de to objekter tyngdekraften yder et arbejde på og $r_1$ og $r_2$ er hhv. start- og slutafstanden mellem de to masser.


\subsubsection{Potentiel energi i et tyngdefelt} \label{afs:tynpot}
Den potentielle energi i et tyngdefelt $U$ på et objekt med masse $m$ kan findes som
\[ 
U = - \frac{GMm}{r}
.\]
Hvor $M$ er massen af objektet som påvirker massen $m$ og $r$ er den indbyrdes afstand mellem massemidtpunktet af massen $M$ og massemidtpunktet af massen $m$.


\subsection{Satellitter i cirkulær bevægelse (13.4)}

\subsubsection{Hastigheden af en satellit i et cirkulært kredsløb} \label{afs:hassat}
For en sattelit i et cirkulært kredsløb om en masse $M$ med en kredsløbsradius på $r$ kan hastigheden af satellitten $v$ findes som
\[ 
v = \sqrt{\frac{GM}{r}}
.\]

\subsubsection{Perioden for en satellit i et cirkulært kredsløb} \label{afs:persat}
For en sattelit med hastighed $v$ i en cirkulær bane med radius $r$ omkring et legeme med masse $M$ kan perioden $T$ findes som
\[ 
T = \frac{2\pi r}{v} = 2\pi r \sqrt{\frac{r}{GM}} = \frac{2\pi r^{\frac{3}{2}}}{\sqrt{GM}}
.\]


\subsection{Keplers love og planeters bevægelse (13.5)}

\subsubsection{Keplers love}
Keplers love lyder
\begin{enumerate}
  \item Alle planeter bevæger sig i elliptiske baner omkring Solen, hvor Solen befinder sig i det ene brændpunkt.
  \item En linje trukket fra en planet til Solen overstryger lige store arealer på lige lange tidsrum.
  \item Kvadratet af en planets omløbstid er proportionalt med kuben af dens middelafstand fra Solen.
\end{enumerate}


\subsubsection{Matematisk formulering af Keplers 2. lov}
Keplers 2. lov kan skrives som
\[ 
\frac{\mathrm{d}A}{\mathrm{d}t} = \frac{1}{2}r^2 \frac{\mathrm{d}\theta}{\mathrm{d}t} 
.\]
Hvor $r$ er radiussen af planetens bane og $\mathrm{d}\theta$ og $\mathrm{d}A$ er hhv. vinklen der tilbagelægges og arealet der overstryges i det lille tidsinterval $\mathrm{d}t$.

\subsubsection{Matematisk formulering af Keplers 3. lov}
Keplers 3. lov kan skrives som
\[ 
T = \frac{2\pi a^{\frac{3}{2}}}{\sqrt{Gm_s}}
.\]
Hvor $T$ er perioden, $a$ er planetens middelafstand fra en sol med massen $m_s$. 



\subsection{Sorte huller (13.8)}

\subsubsection{Swarzschild-radiussen af et sort hul}
Radiussen som en masse $M$ maksimalt må have for at opføre sig som et sort hul betegnes massens Schwarz\-schild-radius $R_s$ og kan findes som
\[ 
R_s = \frac{2GM}{c^2}
.\]
Hvor $c$ er lysets hastighed i et vakuum.
