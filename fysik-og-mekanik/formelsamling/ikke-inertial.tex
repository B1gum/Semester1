\section{Mekanik i ikke-inertial systemer}

\subsection{Acceleration uden rotation}

\subsubsection{Inertialkraftens størrelse}
Inertialkraftens størrelse $F_{\text{inertial}}$ kan findes som
\[ 
F_{\text{inertial}} = -m \Vec{A}
.\]
Hvor $\Vec{A}$ er accelerationen af det accelererende koordinatsystem og $m$ er massen


\subsection{Vinkelhastighedsvektoren}

\subsubsection{Tangentiel hastighed fra radius og vinkelhastighed}
Givet en stedvektor $\Vec{r}$ og en vinkelhastighedsvektor $\Vec{\omega}$ kan den tangentielle hastighedsvektor $\Vec{v_{\text{tan}}}$ findes som
\[ 
\Vec{v_{\text{tan}}} = \Vec{\omega} \times \Vec{r}
.\]

\subsubsection{Sammenhæng mellem afledede i inertial- og ikke-inertialsystemer}
Det gælder at
\[ 
\left( \frac{\mathrm{d} \Vec{Q}}{\mathrm{d}t}  \right)_{S_0} = \left( \frac{\mathrm{d} \Vec{Q}}{\mathrm{d}t}  \right)_{S} + \Vec{\Omega} \times \Vec{Q}
.\]
Hvor $\Vec{Q}$ er en vektor, $\Omega$ er vinkelhastigheden mellem inertialsystemet $S_0$ og ikke-inertialsystemet $S$.


\subsection{Newtons 2. lov for et roterende referencesystem}
Newtons 2. lov for et roterende referencesystem er givet som
\[ 
m \ddot{r} = \Vec{F} + \underbrace{2 m \dot{r} \times \Vec{\Omega}}_{\Vec{F}_{\text{cor}}} + \underbrace{m(\Vec{\Omega} \times \Vec{r}) \times \Omega}_{\Vec{F}_{\text{cf}}}
.\]
Hvor $m$ er massen, $\Vec{r}$ er stedvektoren, $\ddot{r}$ er den anden tidsafledte af $r$, $\dot{r}$ er den første tidsafledte af $r$, $\Vec{F}$ er summen af alle kræfter i det tilsvarende inertialsystem og $\Vec{\Omega}$ er vinkelhastigheden af det roterende referencesystem.


\subsubsection{Korioliskraften}
Korioliskraften $\Vec{F}_{\text{cor}}$ er givet som
\[ 
\Vec{F}_{\text{cor}} = 2m \dot{r} \times \Omega
.\]
Hvor $m$ er massen, $\dot{r}$ er den første tidsafledte af stedvektoren $\Vec{r}$ og $\Vec{\Omega}$ er vinkelhastigheden af det roterende koordinatsystem.


\subsubsection{Centrifugalkraften}
Centrifugalkraften $\Vec{F}_{\text{cf}}$ er givet som
\[ 
\Vec{F}_{\text{cf}} = m(\Vec{\Omega} \times \Vec{r}) \times \Vec{\Omega}
.\]
Hvor $m$ er massen, $\Vec{\Omega}$ er vinkelhastigheden af det roterende koordinatsystem og $\Vec{r}$ er stedvektoren.
