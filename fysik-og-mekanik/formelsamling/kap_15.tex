\section{Mekaniske bølger}

\begin{table}[ht]
\begin{tabular}{|l|l|l|}
\hline
\textbf{Givet} &
  \textbf{Ønsker at finde} &
  \textbf{Relevante formler} \\ \hline
Bølgelænde, frekvens &
  Bølgehastighed &
  \ref{afs:bølhas} \\ \hline
\begin{tabular}[c]{@{}l@{}}Amplitude, vinkelhastighed, position, \\ bølgehastighed, tid\end{tabular} &
  Bølgefunktion &
  \ref{afs:bølfun} \\ \hline
\begin{tabular}[c]{@{}l@{}}Amplitude, position, bølgelængde, tid, \\ periode\end{tabular} &
  Bølgefunktion &
  \ref{afs:bølfun2} \\ \hline
\begin{tabular}[c]{@{}l@{}}Ampltide, bølgetal, position,\\ vinkelfrekvens, tid\end{tabular} &
  Bølgefunktion &
  \ref{afs:bølfun3} \\ \hline
Spænding, masse pr. længde &
  \begin{tabular}[c]{@{}l@{}}Bølgehastighed af trans-\\ versalbølge på en snor\end{tabular} &
  \ref{afs:hastra} \\ \hline
\begin{tabular}[c]{@{}l@{}}Masse pr. længde, spænding, \\ vinkelfrekvens, amplitude\end{tabular} &
  \begin{tabular}[c]{@{}l@{}}Gennemsnitseffekten for\\ sinusoidal bølge på snor\end{tabular} &
  \ref{afs:gnseffbølsno} \\ \hline
Intensitet og to afstande &
  Intensitet til anden afstand &
  \ref{afs:omvkvalov} \\ \hline
\begin{tabular}[c]{@{}l@{}}Ampltiude af stående bølge, bølgetal,\\ position, vinkelfrekvens, tid\end{tabular} &
  \begin{tabular}[c]{@{}l@{}}Bølgefunktion for stående\\ bølge\end{tabular} &
  \ref{afs:bølfunstå} \\ \hline
Bølgehastighed, længde på snor &
  Normalmoder for en snor &
  \ref{afs:normod} \\ \hline
Bølgehastighed, snorens længde &
  Fundamentalfrekvens &
  \ref{afs:funfrek} \\ \hline
Spænding, masse pr. længde &
  Fundamentalfrekvens &
  \ref{afs:funfrek} \\ \hline
\end{tabular}
\end{table}

\subsection{Periodiske bølger og matematiske beskrivelser heraf (15.2-15.3)}

\subsubsection{Bølgehastighed for periodiske bølger} \label{afs:bølhas}
Givet en periodisk bølges bølgelængde $\lambda$ og dens frekvens $f$ kan bølgehastigheden $v$ findes som
\[ 
v = \lambda f
.\]


\subsubsection{Bølgefunktion for en bølge givet vinkel- og bølgehastighed} \label{afs:bølfun}
Bølgefunktionen for en sinusoidal bølge i $+x$-retningen er givet ved
\[ 
y(x,t) = A \cos \left( \omega \left( \frac{x}{v} - t \right) \right)
.\]
Hvor $A$ er amplituden, $\omega$ er vinkelfrekvensen, $x$ er positionen, $v$ er bølgehastigheden og $t$ er tiden.


\subsubsection{Bølgefunktion for en bølge givet bølgelængde og periode} \label{afs:bølfun2}
Bølgefunktionen fra ovenfor kan også skrives som
\[ 
y(x,t) = A \cos \left( 2\pi \left( \frac{x}{\lambda} - \frac{t}{T} \right) \right)
.\]
Hvor $A$ er amplituden, $x$ er positionen, $\lambda$ er bølgelængden, $t$ er tiden og $T$ er perioden.


\subsubsection{Bølgefunktion givet bølgetal og vinkelhastighed} \label{afs:bølfun3}
Bølgefunktionen fra ovenfor kan også skrives som
\[ 
y(x,t) = A \cos (kx - \omega t)
.\]
Hvor $A$ er amplituden, $k$ er bølgetallet ($k = \frac{2\pi}{\lambda}$), $x$ er positionen, $\omega$ er vinkelfrekvensen og $t$ er tiden.


\subsubsection{Bølgeligningen}
Bølgeligningen er
\[ 
\frac{\partial^2 y(x,t)}{\partial x^2} = \frac{1}{v^2} \frac{\partial^2 y(x,t)}{\partial t^2}
.\]
Hvor $\frac{\partial^2 y(x,t)}{\partial x^2}$ er den anden partielt afledede med hensyn til $x$, $v$ er bølgehastigheden of $\frac{\partial^2 y(x,t)}{\partial t^2}$ er den anden partielt afledede med hensyn til $t$.


\subsection{Hastigheden af en transversal bølge (15.4)}

\subsubsection{Hastigheden af en transversal bølge på en snor} \label{afs:hastra}
For en snor med spænding $F$ og masse pr. længde $\mu$ kan bølgehastigheden $v$ findes som
\[ 
v = \sqrt{\frac{F}{\mu}}
.\]


\subsubsection{Hastigheden af en mekanisk bølge}
Hastigheden $v$ af en mekanisk bølge kan findes som
\[ 
v = \sqrt{\frac{\text{Størrelsen på den restaurerende kraft der søger mod at bringe systemet til ligevægt}}{\text{Inertien der forsøger at modstå kraften der søger mod at bringe systemet til ligevægt}}}
.\]


\subsection{Energi i en bølge (15.5)}

\subsubsection{Gennemsnitseffekten for en sinusoidal bølge på en snor} \label{afs:gnseffbølsno}
Gennemsnitseffekten for en sinusoidal bølge på en snor $P_{\text{av}}$ kan findes som
\[ 
P_{\text{av}} = \frac{1}{2}\sqrt{\mu F} \omega^2 A^2
.\]
Hvor $\mu$ er massen pr. længde, $F$ er spændingen i snoren, $\omega$ er vinkelfrekvensen for bølgen og $A$ er bølgens amplitude.


\subsubsection{Den omvendte kvadratlov (Eng: \textit{Inverse square law)}} \label{afs:omvkvalov}
Den omvendte kvadratlov foreskriver en sammenhæng mellem to intensiteter $I$ og to afstande $r$ som
\[ 
\frac{I_1}{I_2} = \frac{r_2^2}{r_1^2}
.\]
Denne lov gælder generelt for bølger der breder sig i 3 dimensioner.


\subsection{Interferens, grænsebetingelser og superposition (15.6)}

\subsubsection{Superpositionsprincippet}
Superpositionsprincippet foreskriver at
\[ 
y(x,t) = y_1(x,t) + y_2(x,t)
.\]
Altså kan summen af to overlappende bølgefunktioner blot adderes for alle punkter for at finde den resulterende kombinerede bølge. 


\subsection{Stående bølger på en snor og en snors normalmoder (15.7-15.8)}

\subsubsection{Bølgefunktionen for en stående bølge på en snor} \label{afs:bølfunstå}
For en stående bølge på en snor med $x=0$-enden fastspændt er bølgefunktionen
\[ 
y(x,t) = (A_{\text{SW}} \sin kx) \sin \omega t
.\]
Hvor $A_{\text{SW}}$ er amplituden af den stående bølge, $k$ er bølgetallet, $x$ er positionen, $\omega$ er vinkelfrekvensen og $t$ er tiden.



\subsubsection{En streng fastspændt i begge enders normalmoder} \label{afs:normod}
Frekvenserne $f_n$ som tilsvarer normalmoderne for en snor kan findes som
\[ 
f_n = n \frac{v}{2L}
.\]
Hvor $n$ er et heltal, $v$ er bølgehastigheden og $L$ er snorens længde.


\subsubsection{Fundamentalfrekvensen for en streng fastspændt i begge ender} \label{afs:funfrek}
Fundamentalfrekvensen for en streng fastspændt i begge ender $f_1$ er normalmoden der tilsvarer $n=1$. Altså
\[ 
f_1 = \frac{v}{2L} = \frac{1}{2L} \sqrt{\frac{F}{\mu}}
.\]
Hvor $v$ er bølgehastigheden, $L$ er snorens længde, $F$ er spændingen i snoren og $\mu$ er snorens masse pr. længde.
