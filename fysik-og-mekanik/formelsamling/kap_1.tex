\section{Enheder, fysiske størrelser og vektorer}

\subsection{Vektorprodukter (1.10)}

\subsubsection{Skalarproduktet}
Lad $\Vec{A}$ og $\Vec{B}$ være to vektorer. Prikproduktet $\Vec{A} \cdot \Vec{B}$ er da defineret som
\[
\Vec{A} \cdot \Vec{B} = AB \cos\phi = \left| \Vec{A} \right| \left| \Vec{B} \right| \cos\phi
\]
Hvor $A$ og $|\Vec{A}|$ er størrelsen af $\Vec{A}$, $B$ og $\Vec{B}$ er størrelsen af $\Vec{B}$ og $\phi$ er vinklen mellem de to vektorer, hvis de lægges så deres ``startpunkter'' er sammenfaldende. For $\ang{0} < \phi < \ang{90}$ er skalarproduktet positivt mens det for $\ang{90} < \phi < \ang{180}$ er negativt -- for $\phi = \ang{90}$ er skalaproduktet 0. 

Skalarproduktet kan også skrives som
\[ 
\Vec{A} \cdot \Vec{B} = A_xB_x + A_yB_y + A_zB_z
.\]
Hvor $\Vec{A} = (A_x, A_y, A_y)$ og $\Vec{B} = (B_x, B_y, B_z)$.

\subsubsection{Krydsproduktet}
Lad $\Vec{A}$ og $\Vec{B}$ være to vektorer. Krydsproduktet $\Vec{A} \times \Vec{B}$ er da defineret som
\[ 
\left| \Vec{C} \right| = \left| \Vec{A} \right| \left| \Vec{B} \right| \sin\phi 
.\]
Hvor $\left| \Vec{C} \right|$ er længden af den resulterende vektor som fås fra krydsproduktet. $\left| \Vec{A} \right|$ og $\left| \Vec{B} \right|$ er længden af hhv. $\Vec{A}$ og $\Vec{B}$ mens $\phi$ er vinklen mellem de to vektorer, hvis de lægges så deres ``startpunkter'' er sammenfaldende. 

Komposanterne til den resulterende vektor af krydsproduktet $\Vec{C} = (C_x, C_y, C_z)$ kan findes som
\[ 
C_x = A_yB_z - A_zB_y, \quad C_y = A_zB_z - A_xB_z, \quad C_z = A_xB_y - A_yB_x
.\]
Hvor $\Vec{A} = (A_x, A_y, A_y)$ og $\Vec{B} = (B_x, B_y, B_z)$.
