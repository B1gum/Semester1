\section{Impuls, kraftimpuls og kollisioner}

\begin{table}[ht]
\begin{tabular}{|l|l|l|}
\hline
\textbf{Givet}                        & \textbf{Ønsker at finde}   & \textbf{Relevante formler} \\ \hline
Masse, hastighed                      & Impuls                     & \ref{afs:imp}              \\ \hline
Impuls                                & Kraft                      & \ref{afs:new2imp}          \\ \hline
Kraft, ændring i tid                  & Kraftimpuls                & \ref{afs:krimpimptheo}     \\ \hline
Ændring i impuls                      & Kraftimpuls                & \ref{afs:krimpimptheo}     \\ \hline
Række af impulser                     & Totalimpuls                & \ref{afs:totimp}           \\ \hline
\begin{tabular}[c]{@{}l@{}}Masse og hastighed af to objekter\\  før inelastisk kollision\end{tabular} &
  \begin{tabular}[c]{@{}l@{}}Samlet hastighed efter\\  inelastisk kollision\end{tabular} &
  \ref{afs:ekonsin} \\ \hline
\begin{tabular}[c]{@{}l@{}}Masse og hastighed af to objekter\\  før elastisk kollision\end{tabular} &
  \begin{tabular}[c]{@{}l@{}}Masse og hastighed af de to \\ objekter efter elastisk kollision\end{tabular} &
  \ref{afs:elakol} \\ \hline
Masser, afstande                      & Massemidtpunkt             & \ref{sec:cm}               \\ \hline
Masser, hastigheder                   & Impuls                     & \ref{afs:cmbev}            \\ \hline
Masse, acceleration af massemidtpunkt & Summen af eksterne kræfter & \ref{afs:new2eks}          \\ \hline
\end{tabular}
\end{table}

\subsection{Impuls og kraftimpuls (8.1)}

\subsubsection{En partikels impuls} \label{afs:imp}
Givet en partikels masse $m$ og dens hastighedsvektor $\Vec{v}$ kan impulsvektoren $\Vec{p}$ findes som
\[ 
\Vec{p} = m \Vec{v}
.\]

\subsubsection{Newtons 2. lov for impuls} \label{afs:new2imp}
Det gælder at
\[ 
\sum \Vec{F} = \frac{\mathrm{d}\Vec{p}}{\mathrm{d}t} 
.\]
Hvor $\Vec{F}$ er kræfterne der virker på partiklen og $\frac{\mathrm{d}\Vec{p}}{\mathrm{d}t}$ er ændringen i impuls over tid.

\subsection{Kraftimpuls-impuls teoeremet} \label{afs:krimpimptheo}
Betragtes en partikel der kun påvirkes af en \textit{konstant} ekstern kraft $\sum \Vec{F}$ over et tidsinterval $\Delta t = t_2 - t_1$ kan kraftimpultsen $\Vec{\jmath}$ af de eksterne kræfter findes som
\[ 
\Vec{\jmath} = \sum \Vec{F} (t_2 - t_1) = \sum \Vec{F} \Delta t
.\]
Det ovenstående kan også skrives som
\[ 
\Vec{\jmath} = \Vec{p_2} - \Vec{p_1} = \Delta \Vec{p}
.\]
Hvor hhv. $\Vec{p_2}$ og $\Vec{p_1}$ er slut- og start-impulsen.

Det ovenstående kan også skrives som
\[ 
\Vec{\jmath} = \int_{t_1}^{t_2}  \sum \Vec{F} \, \mathrm{d}t 
.\]

\subsection{Impulskonservation (8.2)} 
Generelt gælder, at når summen af eksterne kræfter på et system er 0 så er impuls konserveret.

\subsubsection{Totalimpuls for et system af partikler} \label{afs:totimp}
Givet impulsen for alle partikler i et lukket system $\Vec{p_A}, \Vec{p_B}\ldots $ kan den samlede impuls i systemet findes som
\[
\Vec{P} = \Vec{p}_{\text{A}} + \Vec{p}_{\text{B}} + \ldots = m_A \Vec{v}_{\text{A}} + m_B \Vec{v}_{\text{B}} +\ldots 
.\]


\subsection{Impulskonservation og kollisioner (8.3)}
Elastiske kollisioner er alle kollisioner, hvor der ikke tabes mekanisk energi (tænk to billiard- eller marmorkugler der kolliderer). Inelastiske kollisioner er alle kollisioner, hvor den mekaniske energi falder (komplet uealstisk vil medføre at de to kolliderende objekter sidder sammen efter kollisionen).

\subsubsection{Impulskoknservation i en komplet inelastisk kollision} \label{afs:ekonsin}
Eftersom de to objekter sidder sammen efter en inelastisk kollision har vi at
\[ 
  \Vec{v}_{\text{A2}} = \Vec{v}_{\text{B2}} = \Vec{v}_{2}
.\]
Impulskonservation foreskriver da at
\[ 
m_A \Vec{v}_{\text{A1}} + m_B \Vec{v}_{\text{B1}} = (m_A + m_B) \Vec{v}_2
.\]

\subsubsection{Restitutionskoefficienten}
For et stød gælder at resitutionskoefficienten $e$ kan findes som
\[ 
e = - \frac{v_{2f} - v_{1f}}{v_{2i} - v_{1i}}
.\]
Hvor $v_{1i}$ og $v_{1f}$ er hastigheden af objekt 1 hhv. før- og efter kollisionen og $v_{2i}$ og $v_{2f}$ er objekt 2's ditto. For elastiske kollisioner er $e = 1$ og for uelastiske kollisioner er $0< e < 1$, medens $e = 0$ for fuldstædigt uelastiske kollisioner.


\subsection{Elastiske kollisioner (8.4)}  \label{afs:elakol}
For en elastisk kollision i 1 dimension giver konservation af kinetisk energi at
\[ 
\frac{1}{2}m_A v_{A1x}^2 + \frac{1}{2}m_B v_{B1x}^2 = \frac{1}{2}m_A v_{A2x}^2 + \frac{1}{2}m_B v_{B2x}^2
\]
og konservation af impuls giver at
\[ 
m_A v_{A1x} + m_B v_{B1x} = m_A v_{A2x} + m_B v_{B2x}
.\]
Dermed kan sluthastighederne $v_{A2x}$ og $v_{B2x}$ findes såfremt initialhastighederne $v_{A1x}$ og $v_{B1x}$ og masserne $m_A$ og $m_B$ er kendt.


\subsection{Massemidtpunkt (8.5)}

\subsubsection{Massemidtpunktet for et system af partikler} \label{sec:cm}
Er masserne $m_1, m_2, m_3, \ldots$ og stedvektorerne $\Vec{r_1}, \Vec{r_2}, \Vec{r_3}, \ldots $ for en række partikler kendt kan deres fælles massemidtpunkt findes som
\[ 
\Vec{r}_{\text{cm}} = \frac{m_1 \Vec{r}_1 + m_2 \Vec{r}_2 + m_3 \Vec{r}_3 + \ldots }{m_1 + m_2 + m_3 +\ldots} = \frac{\sum_i m_i \Vec{r}_{i}}{\sum_i m_i}
.\]


\subsubsection{Massemidtpunktets bevægelse} \label{afs:cmbev}
Udtrykket fra \ref{sec:cm}: \nameref{sec:cm} kan omskrives til
\[ 
M \Vec{v}_{\text{cm}} = m_1 \Vec{v}_1 + m_2 \Vec{v}_2 + m_3 \Vec{v}_3 + \ldots = \Vec{P}
.\]
Hvor $M$ er den totale masse af alle partiklerne i systemet, $\Vec{v_{cm}}$ er massemidtpunktets hastighedsvektor, $m_1, m_2, m_3,\ldots$ er masserne af de individuelle partikler, $\Vec{v_1}, \Vec{v_2}, \Vec{v_3}, \ldots $ er hastighedsvektorerne til de individuelle partikler og $\Vec{P}$ er systemets totale impuls.


\subsubsection{Eksterne kræfter der virker på en samling af partikler eller et objekt} \label{afs:new2eks}
Det gælder at
\[ 
\sum \Vec{F_{\text{ext}}} = M \Vec{a_{\text{cm}}}
.\]
Hvor $\sum \Vec{F_{\text{ext}}}$ er summen af de eksterne kræfter, der virker på et objekt eller en gruppe af partikler, $M$ er totalmassen af objektet eller gruppen af partikler og $\Vec{a_{\text{cm}}}$ er accelerationen af objektets eller gruppen af partiklers massemidtpunkt.

Altså har vi at når et objekt eller en gruppe af partikler bliver påvirket af en elelr flere eksterne kræfter bevæger massemidtpunktet sig som om al massen var koncentreret netop i det punkt og blev påvirket af en kraft der svarer til summen af alle de eksterne kræfter der virker på systemet.
